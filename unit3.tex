\documentclass[hyperref={pdfpagelabels=false}]{beamer}
%\documentclass[handout]{beamer}
\let\Tiny=\tiny
\hypersetup{pdfpagemode=FullScreen}
\usepackage[ngerman]{babel}
\usepackage[utf8]{inputenc}
\usepackage{graphics}
\usepackage{listings}
\usepackage{verbatim}
%\setbeamertemplate{navigation symbols}{}

\usetheme{Boadilla}

\usecolortheme{beaver}
\usefonttheme{professionalfonts}
\useinnertheme{rounded}
\useoutertheme{smoothbars}
%\useoutertheme{sidebar}

\definecolor{lGray}{gray}{.90}

\newcommand{\code}[1]{\colorbox{lGray}{\texttt{#1}}}
\author{Christian Kniep}

\begin{document}
\title[UNIX Unit 3]{Advanced Operating System with UNIX \\ Unit 3 Processes}  
\institute[ICAT Bandung]{Internation Center of Applied Technologies Bandung}
\date[\today]{\today} 

\begin{frame}
	\titlepage
\end{frame} 

\begin{frame}
	\frametitle{Table of content}
	\tableofcontents
\end{frame} 


\section{3.1 Introduction}
    \subsection{address space \& data structure}
		\begin{frame}
			\frametitle{What is a process?}
			\begin{itemize}
				\item<1-> A process under UNIX consists of an  address space and a set of data structures in the kernel to keep track of that process.
                \item<2-> The memory-section contains:
                \begin{itemize}
                    \item<2-> the code to execute
                    \item<2-> the process stack
                \end{itemize}
                \item<3-> The kernel keeps track of:
                \begin{itemize}
                    \item<3-> adress space map
                    \item<3-> current status of the process
                    \item<3-> execution priority of the process
                    \item<3-> resource usage of the process
                    \item<3-> current signal mask
                    \item<3-> owner of the process
                \end{itemize}
            \end{itemize}
		\end{frame}
    \subsection{what defines a process}
		\begin{frame}
			\frametitle{what you might see}
			\begin{itemize}
				\item<1-> A process has certain attributes, including these:
                \begin{itemize}
                    \item[PID]<1-> The unique identifier of a process
                    \item[PPID]<2-> The parents PID (from whom the process are started)
                    \item[UID]<3-> the UserID of the user who owns the process
                    \item[GID]<4-> the GroupID of the owner
                    \item[EUID]<5-> the EffectiveUserID
                    \item[EGID]<5-> the EffectiveGroupID
                    \item<6-> WTF,what is that suppose to mean!?
                    \item[Priority]<7-> Priority (urgency) the process runs at
                \end{itemize}
            \end{itemize}
		\end{frame}
        \begin{frame}
            \frametitle{The ps-Command}
            \begin{itemize}
                \item<1-> The command \code{ps -l}
                \code{F S   UID   PID  PPID  C PRI  NI ADDR SZ WCHAN  TTY          TIME CMD}
                \code{4 R\ \ 0  6843  6842  0  80   0 -  1065 -  \ \ \  pts/0    00:00:00 bash}
                \code{0 R\ \ 0  7552  6843  0  80   0 -  \ 872 - \ \ \  pts/0    00:00:00 ps}
                \item<2-> \textbf{F} (PROCESS FLAGS)
                \begin{tabular}{ll}
                    \textbf{1}  & forked but didn't exec \\
                    \textbf{4}  & used super-user privileges \\
                    \textbf{30} & according to the book 'loaded into memory'
                \end{tabular}
                \item<3-> \textbf{S} (PROCESS STATE CODES)
                \begin{tabular}{ll}
                    \textbf{D} & Uninterruptible sleep (usually IO) \\
                    \textbf{R} & Running or runnable (on run queue) \\
                    \textbf{S} & Interruptible sleep (waiting for an event to complete) \\
                    \textbf{Z} & Defunct ("zombie") process, terminated but not reaped by its parent.
                \end{tabular}
            \end{itemize}
        \end{frame}
        \begin{frame}
            \frametitle{The ps-Command}
            \begin{itemize}
                \item<1-> \textbf{PID} The ProcessID
                \item<2-> \textbf{PPID} The ParentProcessID
                \item<2-> (\textbf{PGRPID} The ProcessGRouPID)
                \item<3-> \textbf{PRI} The priority of the process
                \item<4-> \textbf{NI} Nice value; used in priority computation 
                \begin{itemize}
                    \item<4-> the lower the value of PRI and NI the higher the priority
                    \item<5-> if a process uses the CPU the PRI-value will raise the nice value
                \end{itemize}
                \item<6-> \textbf{P} Which processor runs the process 
                \item<7-> \textbf{SZ (VSZ)} (Virtual) Memory is used 
                \item<8-> \textbf{TTY} The terminal the process is running at
                \item<9-> \textbf{Time} The cumultative execution time
                \item<10-> \textbf{COMD} The executed command
            \end{itemize}
        \end{frame}
\section{3.2 Processes Priority}
    \subsection{Scheduling}
        \begin{frame}{resources}
            \begin{itemize}
                \item<1-> reward patience
                \begin{itemize}
                    \item<1-> those who have used the least will get access served
                    \item<2-> those who waits for an event will get higher prio (e.g. keyboard)
                \end{itemize}
                \item<3-> The CPU is the most wanted resource. Tied together with the other (such as RAM, Harddrive, Network, etc.) it defines the throughput of an machine
                \item<4-> nice
                \begin{itemize}
                    \item<5-> the nice-command allows the user to manipulate the internal scheduler
                    \item<6-> only a administrator can set a higher priority (lower value)
                    \item<7-> depending on the Unix-flavor the the increments and syntax is different
                \end{itemize}
            \end{itemize}
        \end{frame}
\section{3.3 Other Related Commands}
    \begin{frame}{resources}
        \begin{itemize}
            \item[top]<1-> have a live updating view on the system processes
            \begin{itemize}
                \item<2-> There are many childs of top with different approaches
                \item[htop]<2-> more information with better formating
                \item[ntop]<2-> network-performance (what process does what)
                \item[...]<2-> and so on
            \end{itemize}
            \item[lsof]<3-> to have a look what files are opend (maybe have to be installed first)
            \item[fuser]<4-> what process is using certain files
        \end{itemize}
    \end{frame}
\section{3.4 The fork() System Call}
    \subsection{Explaination}
    \begin{frame}{What is it?}
        \begin{itemize}
            \item<1-> fork() creates a new process
            \item<2-> the memory allocated to the process will be duplicated
            \item<3-> from this line on both processes running the same code
        \end{itemize}
    \end{frame}
\section{3.5 Child Process Termination}
    \subsection{relationships}
    \begin{frame}{Who was first?}
        \begin{itemize}
            \item<1-> if a child dies, the parent has to recognize it
            \item<2-> if the parent don't, the child will bekomm a 'zombie'
        \end{itemize}
    \end{frame}
\section{3.6 Inter Process Communication}
    \begin{frame}{Whats out there?}
        \begin{itemize}
            \item<1-> \textbf{Pipes} Point to point, only one way
            \item<2-> \textbf{Named Pipes} Looks like a file, thing about it like a spot
            \item<3-> \textbf{Message queues} FIFO
            \item<4-> \textbf{Semaphores} Guarded line
            \item<5-> \textbf{Shared Memory} Use the same adresses
            \item<6-> \textbf{Sockets} FIFO in a network
        \end{itemize}
    \end{frame}
    \subsection{3.6.1 Pipes}
    \begin{frame}{Point to Point with FIFO}
        \begin{itemize}
            \item<1-> One-Way communication from one entity to another
            \item<2-> Has an INode but no reference on it, so its \textbf{NOT} a file
            \item<3-> with \code{read} and \code{write} you could use it
        \end{itemize}
    \end{frame}
    \subsection{3.6.2 Named Pipes}
    \begin{frame}{A Pipe With a Name}
        \begin{itemize}
            \item<1-> Like pipes, but both ways and
            \item<2-> they are a file
            \item[$\Rightarrow$]<3-> You have to \code{open()} it first
        \end{itemize}
    \end{frame}
    \subsection{3.6.2 Message Queues}
    \begin{frame}{a bit organized}
        \begin{itemize}
            \item<2-> messages have types (nubmer,msg)
            \item<2-> you could read only type-number X
            \item<3-> could be private (only parent and childs) or public
        \end{itemize}
    \end{frame}
    \subsection{3.6.3 Semaphores}
    \begin{frame}{guard}
        \begin{itemize}
            \item<1-> its like a queue with a counter
            \item<2-> it could seperate enemies
            \item<3-> or unites friends
            \item<4-> its based on a flag and a queue
        \end{itemize}
    \end{frame}
    \subsection{3.6.4 Shared Memory}
    \begin{frame}{on the edge}
        \begin{itemize}
            \item<1-> if you want be fast, why transfer it?
            \item<2-> if you use the same memory-address you don't have to
        \end{itemize}
    \end{frame}
    \subsection{3.6.5 Sockets}
    \begin{frame}{over the network}
        \begin{itemize}
            \item<1-> like named pipes not based on files, but on network-ports
        \end{itemize}
    \end{frame}

    
\end{document}
