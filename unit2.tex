%\documentclass[draft,hyperref={pdfpagelabels=false}]{beamer}
\documentclass[draft,handout]{beamer}
\let\Tiny=\tiny
\hypersetup{pdfpagemode=FullScreen}
\usepackage[ngerman]{babel}
\usepackage[utf8]{inputenc}
\usepackage{graphics}
\usepackage{listings}
\usepackage{verbatim}
%\setbeamertemplate{navigation symbols}{}

\usetheme{Boadilla}

\usecolortheme{beaver}
\usefonttheme{professionalfonts}
\useinnertheme{rounded}
\useoutertheme{smoothbars}
%\useoutertheme{sidebar}

\definecolor{lGray}{gray}{.90}

\newcommand{\code}[1]{\colorbox{lGray}{\texttt{#1}}}
\author{Christian Kniep}

\begin{document}
\title[UNIX]{Advanced Operating System with UNIX}  
\institute[ICAT Bandung]{Internation Center of Applied Technologies Bandung}
\date[27.07.2010]{27. July 2010} 

\begin{frame}
	\titlepage
\end{frame} 

\begin{frame}
	\frametitle{Table of content}
	\tableofcontents
\end{frame} 


\section{Unit 2} 
	\subsection{Introduction to UNIX-Filesystem}
		\begin{frame}
			\frametitle{Introduction}
			\begin{itemize}
				\item<2-> There are only directorys or files, thats it!
                \item<3-> everything is a file, wether it is
                \begin{itemize}
                    \item<4-> a command, textfile, archive, etc.
                    \item<5-> a resources, setting
                \end{itemize}
            \end{itemize}
		\end{frame}
    \subsection{Filesystem}
		\begin{frame}
			\frametitle{In General}
			\begin{itemize}
                \item<2-> The filesystem looks like a tree
				\includegraphics[height=0.5\columnwidth]{pics/fileSystem.png}<2>%
            \end{itemize}
		\end{frame}
		\begin{frame}
			\frametitle{In Detail}
			\begin{itemize}
                \item<2-> The basic folder-hierachie should be
				\includegraphics[height=0.5\columnwidth]{pics/fileSystemSM.png}<2>%
            \end{itemize}
		\end{frame}
	\subsection{Introduction to UNIX-Filesystem}
	    \begin{frame}
			\frametitle{File System Types 1}
			\begin{itemize}
                \item<1-> The Filesystem-Type defines how to use (speak to) the physical device
                \item<1-> There are several ones out there
                \begin{itemize}
                    \item<2-> historical types
                    \begin{itemize}
                        \item<2-> \textbf{s5} for the old SystemV OS
                        \item<2-> \textbf{msdos},\textbf{pcfs} for old versions of DOS / Windows
                    \end{itemize}
                    \item<3-> to use windows-partitions
                    \begin{itemize}
                        \item<3-> \textbf{fat16},\textbf{fat32} the old Windows-Filesystem
                        \item<3-> \textbf{ntfs-3g} to support WindowsXP,Vista and 7
                    \end{itemize}
                \end{itemize}
            \end{itemize}
        \end{frame}
        \begin{frame}
			\frametitle{File System Types 2}
            \begin{itemize}
                    \item<1-> not that common
                    \begin{itemize}
                        \item<1-> \textbf{ufs(2)} for FreeBSD
                        \item<1-> \textbf{bfs} boot file system for SystemV
                    \end{itemize}
                    \item<2-> in broad use
                    \begin{itemize}
                        \item<2-> \textbf{iso9660},\textbf{hsfs} for cdroms
                        \item<2-> \textbf{proc},\textbf{procfs} pseudo-FS in the memory to handle processes 
                        \item<2-> \textbf{ext2},\textbf{ext3} RedHat,debian-derivates
                        \item<2-> \textbf{xfs} SUSE 
                    \end{itemize}
                    \item<3-> upcomming
                    \begin{itemize}
                        \item<3-> \textbf{ext4} due to long history of ext2,ext3
                        \item<3-> \textbf{zfs} pushed by sun, manage a consistant FS
                        \item<3-> \textbf{btfs} speak BetterFS, like ZFS but OpenSource
                    \end{itemize}                
            \end{itemize}
		\end{frame}
    	\begin{frame}
			\frametitle{Swap? Whats that about?}
			\begin{itemize}
                \item<1-> If the memory is fully loaded the OS could outsource some process data to the swap partition
                \item<1-> a drawer (swap) in your desk (RAM) instead of the cabinet (filesystem)
                \item<2-> when the OS uses the swap we say 'the machine is swapping'
                \item<2-> very bad, because its way slower then the normal RAM!
            \end{itemize}
		\end{frame}
    	\begin{frame}
			\frametitle{Fdisk,mkfs}
			\begin{itemize}
                \item<1-> with fdisk you are able to create,alter and delete the partition-table
                \item<1-> new partitions could be set up with a filesystem by using mkfs
            \end{itemize}
		\end{frame}
    \subsection{Mount and Unmount}
	    \begin{frame}
			\frametitle{mount,umount}
			\begin{itemize}
                \item<1-> if you want a device be part of your 'File-System'-tree?
                \item<1-> use \code{mount -t type device mountpoint} to do it
                \item<2-> if you want it to disapear use \code{umount device} or \code{umount mountpoint}
            \end{itemize}
		\end{frame}
    \subsection{Fsck: File System Checking}
	    \begin{frame}
			\frametitle{data errode to your fingertips}
			\begin{itemize}
                \item<1-> power blackout
                \item<1-> 'my dog bites on my pen-drive'
                \item<1-> you name it!
            \end{itemize}
        \end{frame}
        \begin{frame}
			\frametitle{it causes inconsistent states}
            \begin{itemize}
                \item<1-> multiple inodes claimes the same disk block
                \item<1-> a free block is not listed in the superblocks
                \item<1-> a used block is marked free
                \item<1-> ...
            \end{itemize}
		\end{frame}
        \begin{frame}
			\frametitle{soloution 'fsck'}
            \begin{itemize}
                \item<1-> Phase 1 (simple stuff)
                \begin{itemize}
                    \item<1-> Validates the inodes for correctness (format, block numbers)
                    \item<1-> declares blocks BAD (number out of Range), DUP (claimed by another inode)
                \end{itemize}
                \item<2-> Phase 2 (what files/directories are involved?)
                \begin{itemize}
                    \item<2-> Starting from root, \\
                              searches for OUT OF RANGE inode numbers detected in P1
                    \item<2-> found one, than removes the 'dir' or 'file'
                \end{itemize}
            \end{itemize}
        \end{frame}
        \begin{frame}
			\frametitle{soloution 'fsck'}
            \begin{itemize}
                \item<3-> Phase 3 (lost+founds)
                \begin{itemize}
                    \item<3-> Looking for unreferenced directories and stores their files in '$l+f$'
                    \item<3-> the files are named as the inode number
                \end{itemize}
                \item<1-> Phase 4 (check counter)
                \begin{itemize}
                    \item<1-> compares link count information from Phases 2 \& 3, correcting discrepancies
                \end{itemize}
            \end{itemize}
		\end{frame}

    \subsection{The Boot Procedure}
	    \begin{frame}
			\frametitle{Get it on}
			\begin{itemize}
                \item<1-> Hole process from pushing the button to have a login prompt
                \begin{itemize}
                    \item<2-> The memory-resident code
                    \item<2-> self-test
                    \item<3-> probes bus for boot device
                    \item<3-> Reading boot-sector from boot device
                    \item<4-> Boot-program reads kernel and initrd an passs control
                    \item<4-> Kernel identifies,initialise and configure the devices
                    \item<5-> Runs appropriate startup scripts (single- / multi-usermode)
                \end{itemize}
            \end{itemize}
		\end{frame}
	    \begin{frame}
			\frametitle{Get it on}
				\includegraphics[height=0.5\columnwidth]{pics/startUp.png}<1>%
		\end{frame}
    \subsection{Kernel}
	    \begin{frame}
			\frametitle{Startup-process}
            \begin{itemize}
                \item<2-> The kernel ist kind of a puppetmaster who pulls the strings
                \item<2-> If he hadn`t initialise a device, it will not be available
            \end{itemize}
		\end{frame}
    \subsection{System Processes}
	    \begin{frame}
			\frametitle{What comes up?}
			\begin{itemize}
                \item<1-> To start the machine there are various processes that have to run
                \item<1-> the first process called 'swapper' it became PID 0 (but lives not long)
                \item<2-> the 2nd one is the init-Process (PID 1), which forks all the startup-scripts (init-scripts)
                \item<3-> on usual Unix-Systems the PID 2 is the first process created by the init-process and gets the PID 2
            \end{itemize}
		\end{frame}
	    \begin{frame}
			\frametitle{process table}
			\includegraphics[height=0.4\columnwidth]{pics/ps.png}<1>%
		\end{frame}
    \subsection{Startup Scripts}
	    \begin{frame}
			\frametitle{As we are speaking of it...}
			\begin{itemize}
                \item<1-> the init-process (PID 1) are able to start scripts in various ways
                \item<1-> inittab (like in the SMU-book)
                \item<1-> Entry looks like \code{1:2345:respawn:/sbin/getty 38400 tty1}
                \item<2-> The entrys are \code{id:runlevels:action:command}
                \begin{itemize}
                    \item[id] Identifier 
                    \item[runlevel] in which runlevels the command should be started 
                    \item[action] what action should be taken for this command, maybe:
                    \item[commands] specifies the shell command to be run
                \end{itemize}
            \end{itemize}
		\end{frame}
    \subsection{User-/Group-Handling}
	    \begin{frame}
			\frametitle{add Users}
			\begin{itemize}
                \item<1-> if you simply want to add a user you user \code{useradd USERNAME} \\
                          The User will be created without any further Questions. \\
                          Parameters like \code{-G,-g,-s} are possible to add more information
                \item<2-> if you want to be guided you should use \code{adduser USERNAME} \\
                         You will be asked about the Username and some other Information
                \item<3-> To alter user information you use \code{usermod USERNAME}
                \item<4-> \code{userdel USERNAME} deletes the user (-r wipes the home too)
            \end{itemize}
		\end{frame}
	    \begin{frame}
			\frametitle{Groups}
			\begin{itemize}
                \item<1-> same as for users (\code{groupadd}, \code{groupdel},\code{groupmod})
                \item<2-> the information are stored in \code{/etc/groups}
            \end{itemize}
		\end{frame}
	    \begin{frame}
			\frametitle{/etc/passwd}
			\begin{itemize}
                \item<1-> As we said earlier, all settings are stored in files
                \item<1-> So do usersetings: \\
                \scriptsize{
                    $\underbrace{\code{root}}_{\texttt{name}}$:
                    $\underbrace{\code{S3ah8kaR}}_{\texttt{crypt. PW}}$:
                    $\underbrace{\code{0}}_{\texttt{uid}}$:
                    $\underbrace{\code{0}}_{\texttt{gid}}$:
                    $\underbrace{\code{System Administrator}}_{\texttt{description}}$:
                    $\underbrace{\code{/var/root}}_{\texttt{home-dir}}$:
                    $\underbrace{\code{/bin/sh}}_{\texttt{command/shell}}$
                    } \\
                \scriptsize{
                    \code{kniep}:
                    \code{...}:
                    $\underbrace{\code{1000}}_{\texttt{uid > 999}}$:
                    $\underbrace{\code{1000}}_{\texttt{gid > 999}}$:
                    $\underbrace{\code{Christian Kniep,,,}}_{\texttt{GECO-String}}$:
                    \code{/home/kniep}:
                    \code{/bin/bash}
                    }
                \item<1-> Some User-ID explanation :
                \begin{itemize}
                    \item<1-> From start there where 15bit UID ($2^{15}=32768$)
                    \item<1-> modern systems may provide more (now up to 64)
                    \item[id=0]<1-> tied to the root-User                    
                    \item[id\textless100]<1-> reserved for system-user                    
                    \item[id\textless1000]<1-> reserved for users which are running services                    
                    \item[id\textgreater999]<1-> normal useraccounts
                    \item[id=32767]<1-> traditionally the user nobody (opposite to root)
                \end{itemize}
            \end{itemize}
        \end{frame}
    \subsection{Backup \& Restore}
	    \begin{frame}
			\frametitle{you shoud now}
			\begin{itemize}
                \item<1-> since everything is a file you shoud backup the whole
                filesystem, so you will be well prepared for malfunction
                \item<1-> there are a couple of ways to do that:
                \begin{itemize}
                    \item[dump]<1-> according to the book you could use dump,
                                    this will copy blockwise on e.g. a tape
                    \item[tar]<1-> tar is the right choice to zip files/directorys
                    \item[dd]<1-> with (one name-legend said) DiskDump you could copy a data stream.\\
                    This could be a harddisk, or a file, 'you name it!'. The beauty of it is the simplicity and the power. \\
                    You could cut byte out, count it, whatever you want.
                \end{itemize}
            \end{itemize}
		\end{frame}
        \begin{frame}
            \frametitle{additional stuff}
			\begin{itemize}
                \item[split]<1-> do you want to split a large file in many smaller ones?
                \begin{itemize}
                    \item<1-> \code{split -a 1 -l 50 test.txt c} \\
                        This will split the test.txt in files ca,cb,cc,.. containing 50 lines
                    \item<1-> \code{split -a 1 -b 50 m test.img d} \\
                        This will split the test.img in files da,db,dc,.. containing 50MB
                \end{itemize}
                \item[cat]<1-> to bring them back together you use: \\
                            \code{cat c* > test.txt} \\
                            \code{cat d* > test.img} \\
			\end{itemize}
		\end{frame}

\end{document}
